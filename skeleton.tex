\documentclass[12pt,twoside]{article}

%%%%%%%%%%%%%%%%%%%%%%%%%%%%%%%%%%%%%%%%%%%%%%%%%%%%%%%%%%%%%%%%%%%%%%%%%%%%%%%%
% TITLE PAGE INFORMATION
%%%%%%%%%%%%%%%%%%%%%%%%%%%%%%%%%%%%%%%%%%%%%%%%%%%%%%%%%%%%%%%%%%%%%%%%%%%%%%%%

\newcommand{\reporttitle}{Basic Title Page}
\newcommand{\reportauthor}{G. Rozgonyi}
\newcommand{\reportemail}{gergely.rozgonyi20@imperial.ac.uk}
\newcommand{\reportupdated}{2024}

%%%%%%%%%%%%%%%%%%%%%%%%%%%%%%%%%%%%%%%%%%%%%%%%%%%%%%%%%%%%%%%%%%%%%%%%%%%%%%%%
% PATH TO CONFIG FILES
%%%%%%%%%%%%%%%%%%%%%%%%%%%%%%%%%%%%%%%%%%%%%%%%%%%%%%%%%%%%%%%%%%%%%%%%%%%%%%%%

\makeatletter
% ASUS TUF
% \def\input@path{{D:/Documents/OneDrive - Imperial College London/Documents/campus work/.latex_config/}}
% Work Laptop
\def\input@path{{D:/OneDrive - Imperial College London/Documents/campus work/.latex_config/}}
\makeatother

%%%%%%%%%%%%%%%%%%%%%%%%%%%%%%%%%%%%%%%%%%%%%%%%%%%%%%%%%%%%%%%%%%%%%%%%%%%%%%%%
% PAGE LAYOUT
%%%%%%%%%%%%%%%%%%%%%%%%%%%%%%%%%%%%%%%%%%%%%%%%%%%%%%%%%%%%%%%%%%%%%%%%%%%%%%%%

% \usepackage{fancyhdr}
\usepackage[a4paper,hmargin=2.8cm,vmargin=2.0cm]{geometry}

% include files that load packages and define macros
%-------------------------------------------------------------------------------
% PACKAGES
%-------------------------------------------------------------------------------

% checks if compiling with XeTeX
\usepackage{ifxetex}

% for creating graphics programmatically
\usepackage{tikz}

% tikz libraries for path morphing and arrow tips
\usetikzlibrary{
  decorations.pathmorphing, 
  arrows.meta, 
  decorations.markings
}

% for placing text boxes at absolute positions
\usepackage{textpos}

% for bibliographies and citations
\usepackage[numbers]{natbib}
\usepackage{natbib}

% settings for page dimensions and margins, set in main document
% \usepackage[a4paper,hmargin=2.8cm,vmargin=2.0cm,includeheadfoot]{geometry}

% checks if compiling with XeTeX (duplicate)
\usepackage{ifxetex}

% for stacking objects vertically
\usepackage{stackengine}

% for advanced table formatting
\usepackage{tabularx,longtable,multirow,subfigure,caption}%hangcaption

% for customizing label formats
\usepackage{fncylab}

% for customizing headers and footers
\usepackage{fancyhdr}

% for colored text and backgrounds
\usepackage{color}

% for consistent unit spacing and styling
\usepackage[tight,ugly]{units}

% for proper URL formatting
\usepackage{url}

% for improved float (figure/table) positioning
\usepackage{float}

% for multilingual support, here for English
\usepackage[english]{babel}

% for advanced math formatting
\usepackage{amsmath}

% for including and manipulating graphics
\usepackage{graphicx}

% for creating to-do notes with color options
\usepackage[colorinlistoftodos]{todonotes}

% for double stroke maths symbols
\usepackage{amssymb}

% auto-converts .eps graphics to .pdf
\usepackage{epstopdf}

% for adding back-references in bibliographies
\usepackage{backref}

% for additional array and table options
\usepackage{array}

% for additional LaTeX symbols
\usepackage{latexsym}

% for programming tools and macro support
\usepackage{etoolbox}

% for referencing by name instead of number
\usepackage{nameref}

% for "canceling" terms in math expressions
\usepackage{cancel}

% for customizing the enumerate list environment
\usepackage{enumerate}

% for creating colored text boxes
\usepackage{tcolorbox}

% for defining theorem-like environments
\usepackage[standard]{ntheorem}

% for python code formatting
\usepackage{listings}

% for plotting
\usepackage{pgfplots}
\pgfplotsset{compat=1.18}
% for tables
\usepackage{pgfplotstable}
\usepackage{booktabs}

% for pseudocode
\usepackage{algorithm}
\usepackage{algpseudocode}

% for customising enumerate headers
\usepackage{enumitem}

% for better arrows (and possibly else)
\usepackage{mathtools}

% filler text
\usepackage{lipsum}

%-------------------------------------------------------------------------------
% PDF & HYPERREF SETUP
%-------------------------------------------------------------------------------

\ifxetex
% \usepackage{fontspec}
% \setmainfont[Scale=.8]{OpenDyslexic-Regular} % set font style
\else
\usepackage[pdftex,pagebackref,hypertexnames=false,colorlinks]{hyperref} % provide links in pdf
\hypersetup{pdftitle={},
  pdfsubject={}, 
  pdfauthor={\reportauthor},
  pdfkeywords={}, 
  pdfstartview=FitH,
  pdfpagemode={UseOutlines},% None, FullScreen, UseOutlines
  bookmarksnumbered=true, bookmarksopen=true, colorlinks,
    citecolor=black,%
    filecolor=black,%
    linkcolor=black,%
    urlcolor=black}
\usepackage[all]{hypcap}
\fi

%-------------------------------------------------------------------------------
% PARAGRAPH BREAKS
%-------------------------------------------------------------------------------

\setlength{\parindent}{0em}  % indentation of paragraph

\setlength{\parskip}{1.5ex plus 0.5ex minus 0.5ex} % spacing between paragraphs

%-------------------------------------------------------------------------------
% HEADER & FOOTER
%-------------------------------------------------------------------------------

% callable current section name in small caps
\makeatletter
\newcommand*{\currentname}{%
\ifnum\value{subsection}=0
  \thesection. \textsc{\@currentlabelname} % Only section
\else
  \thesubsection. \textsc{\@currentlabelname} % Subsection
\fi
}
\makeatother

\pagestyle{fancy}

\fancyhf{} % clear all header and footer fields

\fancyhead[L]{\myheaderleft}
\fancyhead[R]{\myheaderright}
\fancyhead[C]{\myheadercenter}

\fancyfoot[ER,OL]{\thepage}

\fancyfoot[OC,EC]{} % center footer

\renewcommand{\headrulewidth}{0.1pt} % under the header

\renewcommand{\footrulewidth}{0.1pt} % above the footer

% Define commands for header content
\newcommand{\myheaderleft}{} % default value
\newcommand{\myheaderright}{}
\newcommand{\myheadercenter}{\currentname}

%-------------------------------------------------------------------------------
% FIGURE & TABLE CAPTIONS
%-------------------------------------------------------------------------------

\captionsetup{margin=6pt,font=small,labelfont=bf}

%-------------------------------------------------------------------------------
% CHAPTER HEADINGS
%-------------------------------------------------------------------------------

\def\@makechapterhead#1{%
  \vspace*{10\p@}% vertical space above the chapter title
  {\parindent \z@ \raggedright % no paragraph indentation and text to the left
        % add 'Chapter' and its number in uppercase
        {\Large \MakeUppercase{\@chapapp} \space \thechapter}
        \\ % line break
        \hrulefill % add a horizontal line
        \par\nobreak % new paragraph without allowing a page break
    \interlinepenalty\@M % penalty discourage page breaks within a line
    \Huge \bfseries % font size huge and bold
    % print the chapter number and title, then prevents a page break
    \thechapter \space\space #1\par\nobreak
    \vskip 30\p@ % vertical space after the chapter title
  }
}

% Definition of the command for chapter headings without numbering (e.g., \chapter*)
\def\@makeschapterhead#1{%
  \vspace*{10\p@} % vertical space above the chapter title
  {\parindent \z@ \raggedright % no paragraph indentation and text to the left
    \interlinepenalty\@M % discourage page breaks within a line
    \Huge \bfseries % font size huge and bold
    % print the chapter number and title, then prevents a page break
    #1\par\nobreak
    \vskip 30\p@ % vertical space after the chapter title
  }
}

%-------------------------------------------------------------------------------
% BOXIT ENVIRRONENT
%-------------------------------------------------------------------------------

% Define the beginning of the boxit environment
\def\Beginboxit
   {\par % new paragraph
    \vbox\bgroup % begin vertical box and a group to contain the box
	   \hrule % horizontal line at the top of the box
	   \hbox\bgroup % begin horizontal box to contain the content
		  \vrule \kern1.2pt % vertical line on the left small space
		  \vbox\bgroup\kern1.2pt % start another box for content, small space
   }

% Define the end of the boxit environment
\def\Endboxit{%
			      \kern1.2pt % small space at the end of the content
		       \egroup % end inner vertical box
		  \kern1.2pt\vrule % small space and vertical line on the right
		\egroup % end the horizontal box
	   \hrule % horizontal line at the bottom of the box
	 \egroup % end the outermost vertical box
   }

% environment 'boxit' using Beginboxit and Endboxit
\newenvironment{boxit}{\Beginboxit}{\Endboxit}

% 'boxit*' stretches horizontally to fill the page width
\newenvironment{boxit*}{\Beginboxit\hbox to\hsize{}}{\Endboxit}

%-------------------------------------------------------------------------------
% EQUATIONS
%-------------------------------------------------------------------------------

\allowdisplaybreaks % allow page breaks in equations

\setlength{\jot}{8pt} % spacing between equations

%-------------------------------------------------------------------------------
% PYTHON CODE FORMATTING
%-------------------------------------------------------------------------------

% Define custom colors
\definecolor{codegreen}{rgb}{0,0.6,0}
\definecolor{codegray}{rgb}{0.5,0.5,0.5}
\definecolor{codepurple}{rgb}{0.58,0,0.82}
\definecolor{backcolour}{rgb}{0.95,0.95,0.92}

% Setup the style for Python
\lstdefinestyle{pythonstyle}{
    backgroundcolor=\color{backcolour},   
    commentstyle=\color{codegreen},
    keywordstyle=\color{magenta},
    numberstyle=\color{codegray},
    stringstyle=\color{codepurple},
    basicstyle=\ttfamily,
    breakatwhitespace=false,         
    breaklines=true,                 
    captionpos=b,                    
    keepspaces=true,                 
    numbers=left,                    
    numbersep=5pt,                  
    showspaces=false,                
    showstringspaces=false,
    showtabs=false,                  
    tabsize=2
}

\lstset{
  breaklines=true,
  style=pythonstyle
}

\renewcommand{\lstlistingname}{Code Snippet}

% use this to include a Python script from a file
% \lstinputlisting[language=Python, caption=Python script from file]{script.py}

%-------------------------------------------------------------------------------
% SECTION COUNTING FOR STARRED VARIANTS
%-------------------------------------------------------------------------------

\newcommand{\sections}[1]{
  \stepcounter{section}
  \addcontentsline{toc}{section}{#1}
}
\newcommand{\subsections}[1]{
  \stepcounter{subsection}
  \addcontentsline{toc}{subsection}{#1}
}
\newcommand{\subsubsections}[1]{
  \stepcounter{subsubsection}
  \addcontentsline{toc}{subsubsection}{#1}
} % various packages needed for maths etc.
% vectors
\renewcommand{\vec}[1]{{\boldsymbol{{#1}}}}
\newcommand{\mat}[1]{{\boldsymbol{{#1}}}}

%%% various color definitions
\definecolor{darkgreen}{rgb}{0,0.6,0}

\newcommand{\blue}[1]{{\color{blue}#1}}
\newcommand{\red}[1]{{\color{red}#1}}
\newcommand{\green}[1]{{\color{darkgreen}#1}}
\newcommand{\orange}[1]{{\color{orange}#1}}
\newcommand{\magenta}[1]{{\color{magenta}#1}}
\newcommand{\cyan}[1]{{\color{cyan}#1}}

% redefine emph to be bold blue
\renewcommand{\emph}[1]{\blue{\bf{#1}}}

% red circle around a character
\newcommand*\circled[1]{\tikz[baseline=(char.base)]{
            \node[shape=circle,draw=red,inner sep=0.5pt] (char) {#1};}}
% blue square around a character
\newcommand*\squared[1]{\tikz[baseline=(char.base)]{
            \node[shape=rectangle,draw=blue,inner sep=3pt] (char) {#1};}}
% shield symbol
\newcommand\shield{
    \tikz [baseline] \draw[line width=0.15ex] (0,1.3ex) 
    -- (0,0.65ex) 
    arc [radius=0.5ex, start angle=-180, end angle=0] 
    -- (1ex,1.3ex) 
    -- cycle;
}

% real and imaginary parts
\renewcommand{\Re}{\operatorname{Re}}
\renewcommand{\Im}{\operatorname{Im}} % short-hand notation and macros

\fancyhf{}
\renewcommand{\headrulewidth}{0pt} % under the header
\renewcommand{\footrulewidth}{0pt} % above the footer

% can comment out if you don't want page numbering
\fancyfoot[C]{\thepage} 

%%%%%%%%%%%%%%%%%%%%%%%%%%%%%%%%%%%%%%%%%%%%%%%%%%%%%%%%%%%%%%%%%%%%%%%%%%%%%%%%
% ABSTRACT
%%%%%%%%%%%%%%%%%%%%%%%%%%%%%%%%%%%%%%%%%%%%%%%%%%%%%%%%%%%%%%%%%%%%%%%%%%%%%%%%

\newcommand{\reportabstract}{
    This is a simple document with a title page. Your abstract would go here
}




%%%%%%%%%%%%%%%%%%%%%%%%%%%%%%%%%%%%%%%%%%%%%%%%%%%%%%%%%%%%%%%%%%%%%%%%%%%%%%%%
% DOCUMENT
%%%%%%%%%%%%%%%%%%%%%%%%%%%%%%%%%%%%%%%%%%%%%%%%%%%%%%%%%%%%%%%%%%%%%%%%%%%%%%%%

\begin{document}

%%%%%%%%%%%%%%%%%%%%%%%%%%%%%%%%%%%%%%%%%%%%%%%%%%%%%%%%%%%%%%%%%%%%%%%%%%%%%%%%
% Title Page (can comment out if not needed)
%%%%%%%%%%%%%%%%%%%%%%%%%%%%%%%%%%%%%%%%%%%%%%%%%%%%%%%%%%%%%%%%%%%%%%%%%%%%%%%%

\begin{titlepage}

    \begin{center}
        \vspace*{4cm}
        
        \Huge
        \reporttitle
        
        \vspace{1cm}
        \LARGE
        \reportauthor
         
        \vspace{0.5cm}
        
        \LARGE
        UPDATED IN \reportupdated
        
        \vspace{1.5cm}
        
        \large
        Department of Mathematics\\
        Imperial College London\\
        180 Queen’s Gate\\
        London, SW7 2AZ, U.K.\\
        
        \vspace{0.5cm}
        
        \texttt{\href{mailto:\reportemail}{\reportemail}}
        
        \vspace{2cm}
        
        \large
        \textbf{Abstract}
        
        \vspace{0.5cm}

        \normalsize
        {\reportabstract}

        \vfill
        
    \end{center}
    
\end{titlepage}

\setcounter{page}{1}

%%%%%%%%%%%%%%%%%%%%%%%%%%%%%%%%%%%%%%%%%%%%%%%%%%%%%%%%%%%%%%%%%%%%%%%%%%%%%%%%
% Main Document
%%%%%%%%%%%%%%%%%%%%%%%%%%%%%%%%%%%%%%%%%%%%%%%%%%%%%%%%%%%%%%%%%%%%%%%%%%%%%%%%

\section*{Basic LaTeX functionalities}

This document is as plain as it gets. But you do have page numbering.

Here's an equation:
\begin{equation}
    \int_{0}^{\infty} e^{-x^2} dx = \frac{\sqrt{\pi}}{2}
    \label{eq:gaussian_integral}
\end{equation}

And here's a table:
\begin{table}[H]
    \centering
    \begin{tabular}{c|c}
        $x$ & $f(x)$ \\
        \hline
        1 & 2 \\
        2 & 4 \\
        3 & 6 \\
        4 & 8 \\
    \end{tabular}
    \caption{A simple table}
    \label{tab:simple_table}
\end{table}
Notice that it's positioned to be exactly here using the \texttt{[H]} in the
beginning line. Otherwise, it would float to the top of the page (or the next
suitable page).

The code for the Imperial logo comes here, but it floats to the top of the page.
Notice that it's set to be 60\% of the width of the writable page 
(\texttt{[width=0.6\textbackslash textwidth]}).
\begin{figure}
    \centering
    \includegraphics[width=0.6\textwidth]{./figures/imperial.pdf}
    \caption{Imperial College London logo}
    \label{fig:imperial_logo}
\end{figure}

I can reference each of these elements using the \texttt{label} command. For
example, I can reference the table as Table \ref{tab:simple_table}, the equation
as Equation \ref{eq:gaussian_integral}, and the figure as Figure
\ref{fig:imperial_logo}.

If I want to include a multiline equation, I can use the \texttt{align} 
environment:
\begin{align}
    1+1+1+1+1&=2+1+1+1 \nonumber\\
    &=3+1+1 \\
    &=4+1 \nonumber\\
    &=5
\end{align}
I put \texttt{\textbackslash nonumber} in the lines I don't want to be numbered.

To include code, I can use the \texttt{lstlisting} environment. But this can
go into a new \texttt{\textbackslash subsection}, that I decide to number:

\subsection{Including code}

Wait! The number is all weird. I'd prefer it to be \texttt{1.1} instead of
\texttt{0.1}. Let's fix this by telling LaTeX that we are already in the first
section, using the \texttt{stepcounter} command.
\stepcounter{section}
But we also want to reset the subsection counter to 0. We can do this using the
\texttt{setcounter} command.
\setcounter{subsection}{0}
Let's see if this worked.

\subsection{Including code}

Yay! Onto displaying code.

\begin{lstlisting}[language=Python, caption=Python example]
# This is a comment
import numpy as np

def inc(x):
    return x + 1

print(inc(4))
\end{lstlisting}
This can be used to include code from any language. I can include C code:
\begin{lstlisting}[language=C]
#include <iostream>

int main() {
    std::cout << "Hello, world!" << std::endl;
    return 0;
}
\end{lstlisting}
and (for whatever reason) I even removed the caption!

I can also include code from a file:
\lstinputlisting[language=Python, caption=Python example from file]{
    ./code/example.py
}

Now that's done, let's draw something using TikZ.

\subsection{Drawing with TikZ}

\begin{figure}[H]
    \centering
    \begin{tikzpicture}

        \draw[fill=yellow] (0,0) circle (1);
        \draw (0,0) circle (1);
        \draw (-0.5,0.5) circle (0.1);
        \draw (0.5,0.5) circle (0.1);
        \draw (-0.5,-0.5) -- (0.5,-0.5);

    \end{tikzpicture}
    \caption{A simple vector graphic face.}
    \label{fig:tikz_2d}
\end{figure}

But maaybe you generated some data and saved it as a \texttt{.csv} file. You can
include a plot of this too!

\begin{figure}[H]
    \centering
    \begin{tikzpicture}
        \begin{axis}[
            title={Data Plot},
            xlabel={$x$},
            ylabel={$f(x)$},
            legend pos=north west,
            ymajorgrids=true,
            grid style=dashed,
        ]
        \addplot table [x=x, y=y, col sep=comma] {./code/data.csv};
        \addlegendentry{Data}
        \end{axis}
    \end{tikzpicture}
    \caption{A simple plot of some data.}
    \label{fig:tikz_plot}
\end{figure}

Finally, some useful LaTeX commands and maths stuff, like matrices:
\begin{itemize}
    \item \texttt{\textbackslash textbf\{bold\}}: \textbf{bold}
    \item \texttt{\textbackslash textit\{italic\}}: \textit{italic}
\end{itemize}

\begin{equation}
    \begin{bmatrix}
        1 & 2 & 3 \\
        4 & 5 & 6 \\
        7 & 8 & 9
    \end{bmatrix}=
    \begin{pmatrix}
        1 & 2 & 3 \\
        4 & 5 & 6 \\
        7 & 8 & 9
    \end{pmatrix}
\end{equation}

\end{document}
