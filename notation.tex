% vectors
\renewcommand{\vec}[1]{{\boldsymbol{{#1}}}}
\newcommand{\mat}[1]{{\boldsymbol{{#1}}}}

%%% various color definitions
\definecolor{darkgreen}{rgb}{0,0.6,0}

\newcommand{\blue}[1]{{\color{blue}#1}}
\newcommand{\red}[1]{{\color{red}#1}}
\newcommand{\green}[1]{{\color{darkgreen}#1}}
\newcommand{\orange}[1]{{\color{orange}#1}}
\newcommand{\magenta}[1]{{\color{magenta}#1}}
\newcommand{\cyan}[1]{{\color{cyan}#1}}

% redefine emph to be bold blue
\renewcommand{\emph}[1]{\blue{\bf{#1}}}

% red circle around a character
\newcommand*\circled[1]{\tikz[baseline=(char.base)]{
            \node[shape=circle,draw=red,inner sep=1.3pt] (char) {#1};}}
% blue square around a character
\newcommand*\squared[1]{\tikz[baseline=(char.base)]{
            \node[shape=rectangle,draw=blue,inner sep=3pt] (char) {#1};}}
% shield symbol
\newcommand\shield{
    \tikz [baseline] \draw[line width=0.15ex] (0,1.3ex) 
    -- (0,0.65ex) 
    arc [radius=0.5ex, start angle=-180, end angle=0] 
    -- (1ex,1.3ex) 
    -- cycle;
}

% real and imaginary parts
\renewcommand{\Re}{\operatorname{Re}}
\renewcommand{\Im}{\operatorname{Im}}


% partial and total derivatives
\NewDocumentCommand{\pd}{omm}{
    \IfValueTF{#1}{
        \frac{\partial^{#1} #2}{\partial #3^{#1}}
    }{
        \frac{\partial #2}{\partial #3}
    }
}

\NewDocumentCommand{\td}{omm}{
    \IfValueTF{#1}{
        \frac{d^{#1} #2}{d #3^{#1}}
    }{
        \frac{d #2}{d #3}
    }
}

% argmin and argmax
\DeclareMathOperator*{\argmax}{argmax}
\DeclareMathOperator*{\argmin}{argmin}

\newcommand{\etal}{\textit{et al. }}