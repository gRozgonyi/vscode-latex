%-------------------------------------------------------------------------------
% PACKAGES
%-------------------------------------------------------------------------------

% checks if compiling with XeTeX
\usepackage{ifxetex}

% for creating graphics programmatically
\usepackage{tikz}

% tikz libraries for path morphing and arrow tips
\usetikzlibrary{decorations.pathmorphing, arrows.meta}

% for placing text boxes at absolute positions
\usepackage{textpos}

% for bibliographies and citations
\usepackage{natbib}

% settings for page dimensions and margins, set in main document
% \usepackage[a4paper,hmargin=2.8cm,vmargin=2.0cm,includeheadfoot]{geometry}

% checks if compiling with XeTeX (duplicate)
\usepackage{ifxetex}

% for stacking objects vertically
\usepackage{stackengine}

% for advanced table formatting
\usepackage{tabularx,longtable,multirow,subfigure,caption}%hangcaption

% for customizing label formats
\usepackage{fncylab}

% for customizing headers and footers
\usepackage{fancyhdr}

% for colored text and backgrounds
\usepackage{color}

% for consistent unit spacing and styling
\usepackage[tight,ugly]{units}

% for proper URL formatting
\usepackage{url}

% for improved float (figure/table) positioning
\usepackage{float}

% for multilingual support, here for English
\usepackage[english]{babel}

% for advanced math formatting
\usepackage{amsmath}

% for including and manipulating graphics
\usepackage{graphicx}

% for creating to-do notes with color options
\usepackage[colorinlistoftodos]{todonotes}

% for double stroke fonts, often in math
\usepackage{dsfont}

% auto-converts .eps graphics to .pdf
\usepackage{epstopdf}

% for adding back-references in bibliographies
\usepackage{backref}

% for additional array and table options
\usepackage{array}

% for additional LaTeX symbols
\usepackage{latexsym}

% for programming tools and macro support
\usepackage{etoolbox}

% for referencing by name instead of number
\usepackage{nameref}

% for "canceling" terms in math expressions
\usepackage{cancel}

% for customizing the enumerate list environment
\usepackage{enumerate}

% for creating colored text boxes
\usepackage{tcolorbox}

% for defining theorem-like environments
\usepackage{ntheorem}

% for python code formatting
\usepackage{listings}

%-------------------------------------------------------------------------------
% PDF & HYPERREF SETUP
%-------------------------------------------------------------------------------

\ifxetex
% \usepackage{fontspec}
% \setmainfont[Scale=.8]{OpenDyslexic-Regular} % set font style
\else
\usepackage[pdftex,pagebackref,hypertexnames=false,colorlinks]{hyperref} % provide links in pdf
\hypersetup{pdftitle={},
  pdfsubject={}, 
  pdfauthor={\reportauthor},
  pdfkeywords={}, 
  pdfstartview=FitH,
  pdfpagemode={UseOutlines},% None, FullScreen, UseOutlines
  bookmarksnumbered=true, bookmarksopen=true, colorlinks,
    citecolor=black,%
    filecolor=black,%
    linkcolor=black,%
    urlcolor=black}
\usepackage[all]{hypcap}
\fi

%-------------------------------------------------------------------------------
% THEOREM SETUP
%-------------------------------------------------------------------------------

\theoremstyle{break}
\newtheorem{lemma}{Lemma}
\newtheorem{theorem}{Theorem}
\newtheorem*{remark}{Remark}
\newtheorem{definition}{Definition}
\newtheorem*{proposition}{Proposition}

%-------------------------------------------------------------------------------
% PARAGRAPH BREAKS
%-------------------------------------------------------------------------------

\setlength{\parindent}{0em}  % indentation of paragraph

\setlength{\parskip}{1.5ex plus 0.5ex minus 0.5ex} % spacing between paragraphs

%-------------------------------------------------------------------------------
% HEADER & FOOTER
%-------------------------------------------------------------------------------

% callable current section name in small caps
\makeatletter
\newcommand*{\currentname}{\thesection. \textsc{\@currentlabelname}}
\makeatother

\pagestyle{fancy}

\fancyhf{} % clear all header and footer fields

\fancyhead[L]{\myheaderleft}
\fancyhead[R]{\myheaderright}
\fancyhead[C]{\myheadercenter}

\fancyfoot[ER,OL]{\thepage}

\fancyfoot[OC,EC]{} % center footer

\renewcommand{\headrulewidth}{0.1pt} % under the header

\renewcommand{\footrulewidth}{0.1pt} % above the footer

% Define commands for header content
\newcommand{\myheaderleft}{} % default value
\newcommand{\myheaderright}{}
\newcommand{\myheadercenter}{\currentname}

%-------------------------------------------------------------------------------
% FIGURE & TABLE CAPTIONS
%-------------------------------------------------------------------------------

\captionsetup{margin=6pt,font=small,labelfont=bf}

%-------------------------------------------------------------------------------
% CHAPTER HEADINGS
%-------------------------------------------------------------------------------

\def\@makechapterhead#1{%
  \vspace*{10\p@}% vertical space above the chapter title
  {\parindent \z@ \raggedright % no paragraph indentation and text to the left
        % add 'Chapter' and its number in uppercase
        {\Large \MakeUppercase{\@chapapp} \space \thechapter}
        \\ % line break
        \hrulefill % add a horizontal line
        \par\nobreak % new paragraph without allowing a page break
    \interlinepenalty\@M % penalty discourage page breaks within a line
    \Huge \bfseries % font size huge and bold
    % print the chapter number and title, then prevents a page break
    \thechapter \space\space #1\par\nobreak
    \vskip 30\p@ % vertical space after the chapter title
  }
}

% Definition of the command for chapter headings without numbering (e.g., \chapter*)
\def\@makeschapterhead#1{%
  \vspace*{10\p@} % vertical space above the chapter title
  {\parindent \z@ \raggedright % no paragraph indentation and text to the left
    \interlinepenalty\@M % discourage page breaks within a line
    \Huge \bfseries % font size huge and bold
    % print the chapter number and title, then prevents a page break
    #1\par\nobreak
    \vskip 30\p@ % vertical space after the chapter title
  }
}

%-------------------------------------------------------------------------------
% BOXIT ENVIRRONENT
%-------------------------------------------------------------------------------

% Define the beginning of the boxit environment
\def\Beginboxit
   {\par % new paragraph
    \vbox\bgroup % begin vertical box and a group to contain the box
	   \hrule % horizontal line at the top of the box
	   \hbox\bgroup % begin horizontal box to contain the content
		  \vrule \kern1.2pt % vertical line on the left small space
		  \vbox\bgroup\kern1.2pt % start another box for content, small space
   }

% Define the end of the boxit environment
\def\Endboxit{%
			      \kern1.2pt % small space at the end of the content
		       \egroup % end inner vertical box
		  \kern1.2pt\vrule % small space and vertical line on the right
		\egroup % end the horizontal box
	   \hrule % horizontal line at the bottom of the box
	 \egroup % end the outermost vertical box
   }	

% environment 'boxit' using Beginboxit and Endboxit
\newenvironment{boxit}{\Beginboxit}{\Endboxit}

% 'boxit*' stretches horizontally to fill the page width
\newenvironment{boxit*}{\Beginboxit\hbox to\hsize{}}{\Endboxit}

%-------------------------------------------------------------------------------
% EQUATIONS
%-------------------------------------------------------------------------------

\allowdisplaybreaks % allow page breaks in equations

\setlength{\jot}{8pt} % spacing between equations

%-------------------------------------------------------------------------------
% PYTHON CODE FORMATTING
%-------------------------------------------------------------------------------

% Define custom colors
\definecolor{codegreen}{rgb}{0,0.6,0}
\definecolor{codegray}{rgb}{0.5,0.5,0.5}
\definecolor{codepurple}{rgb}{0.58,0,0.82}
\definecolor{backcolour}{rgb}{0.95,0.95,0.92}

% Setup the style for Python
\lstdefinestyle{pythonstyle}{
    backgroundcolor=\color{backcolour},   
    commentstyle=\color{codegreen},
    keywordstyle=\color{magenta},
    numberstyle=\color{codegray},
    stringstyle=\color{codepurple},
    basicstyle=\ttfamily,
    breakatwhitespace=false,         
    breaklines=true,                 
    captionpos=b,                    
    keepspaces=true,                 
    numbers=left,                    
    numbersep=5pt,                  
    showspaces=false,                
    showstringspaces=false,
    showtabs=false,                  
    tabsize=2
}

\lstset{
  breaklines=true,
  style=pythonstyle
}

\renewcommand{\lstlistingname}{Code Snippet}

% use this to include a Python script from a file
% \lstinputlisting[language=Python, caption=Python script from file]{script.py}